\documentclass[11.5pt, twoside, a4paper]{article}
\usepackage{graphicx, amssymb, amsmath, amsthm, xfrac, mathabx, upgreek, fancyhdr, float}


\begin{document}

\title{Project Proposal: Underwater Robot for Fish Farms}
\author{Chanelle Lee}
\date{\today}
\maketitle

\begin{abstract}
ABSTRACT
\end{abstract}

\section{Wider Goals}
This project aims to be part of wider work towards a prototype robotic platform for inspecting and cleaning nets underwater on commercial fish farms. Specifically, this robotic platform would be able to;
\begin{enumerate}
\item move autonomously about the net,
\item perform net inspection tasks,
\item clean the net of fouling, and
\item monitor fish.
\end{enumerate}
The timescale of this project is insufficient to tackle all of these problems and so the second has been chosen as the main aim of this project, with some brief discussion of the mechanism of movement about the net as such as it impacts upon the main aim.

\section{Aims and Objectives}
As discussed this project will concentrate the performance of net inspection tasks of an underwater net cleaning robotic platform. The three aims of these inspection tasks, and the project objectives for each, are:
\begin{enumerate}
\item Identify damage to the net.
\begin{itemize}
\item Study image processing techniques for locating a net within an underwater image.
\item Design an algorithm for detecting damage to the net.
\end{itemize}
\item Identify and quantify biofouling on the net.
\begin{itemize}
\item Study methods of identifying biofouling. %need to be more specific here
\item Study methods of quantifying biofouling.
\item Create software to identify biofouling within the image, and quantify the magnitude.
\end{itemize}
\item Recognise specific species of biofouling on the net.
\begin{itemize}
\item Study visual differences between types of biofouling, with emphasis on more harmful types.
\item Study classification methods.
\item Create software to recognise different species, specifically warning if recognise harmful species.
\end{itemize}
\end{enumerate}
All of these aims will be tested against images from fish farms and real net in a tank, where the levels of biofouling and the quality of the water will be varied to test performance.


\section{Motivation}

Aquaculture is an increasingly important industry to Scotland, helping to create `sustainable economic growth in rural and coastal communities' \cite{ScotAqua} and is the world's third largest supplier of Atlantic Salmon. One important challenge to be tackled in aquaculture is the issue of biofouling; the colonisation of marine algae and animals on submerged components of cages and nets \cite{fitridge2012impact}. Biofouling of nets and cages has four main harmful effects \cite{fitridge2012impact} \cite{beveridge2008cage} \cite{Crown}:
\begin{itemize}
\item Restriction of water exchange from the occlusion of the netting mesh, which leads to lower dissolved oxygen (DO) levels and a build up of excess feed and waste. 
\item Cage deformation and structural fatigue from the extra weight and hydrodynamic forces upon the net.
\item Disease risk as fouling communities can act as reservoirs for pathogens and parasites.
\item The escape of stock due to net damage.
\end{itemize}
These all have a significant impact on the profitability of aquaculture through the loss of stock and net replacement and its removal is `expensive and labour-intensive, accounting for $20+\%$ of the market price of produce' \cite{Crown}.

Current methods of biofouling removal include manual cleaning both in and out of the water, chemical antifoulants and biological control. Manual cleaning of nets usually requires their removal from the water and then manual scrubbing with brushes and pressurized water; however, not only is this `tedious and labour intensive' \cite{braithwaite2004marine}, but removing the nets from the water can cause damage to the nets and stress the fish by interrupting feeding schedules, hence affecting growth rates and even leading to mortalities \cite{Crown}. In situ net cleaning is traditionally done manually done with divers, which although mitigates the detrimental effects on stock, is still expensive and labour-intensive, but now `underwater cleaners are in widespread use' and `in situ cleaning is now almost fully automated and the dominant removal strategy in the largest fish farms' \cite{fitridge2012impact}. 












\section{Literature Review}

\section{Risk Register}

\section{Gantt Chart}

\bibliographystyle{unsrt}
\bibliography{FishProject}{}


\end{document}